\chapter{Introduction} \label{Introduction}
\todo*{5-10\%, including motivation, general audience}

\pagenumbering{arabic}
In the last couple of years gamification has found it's way into many areas of our daily life. In regard to our personal life, companies like Amazon or Runtastic can base their gamification approach on publicly sharing personal achievements and statistics to improve user commitment. In contrast, gamification concerning our work life can have much higher privacy demands. Since comparison is a key component for the gamification approach, privacy protecting computations of system wide statistical values (for example minimum and maximum) are needed. The solution comes in the form of \gls{SMPC}, a subfield of cryptography.

Existing frameworks for \gls{SMPC} utilize the Internet protocol, though access to the Internet or even a \gls{LAN} cannot be provided in all environments. Especially many hospitals tend to avoid Wi-Fi to reduce the risk of electromagnetic interference with medical devices.

To be able to utilize \gls{SMPC} in environments with Wi-Fi restrictions, this thesis studies the characteristics of mesh-networks and proposes describes the design of a \gls{SMPC} framework for mesh-networks.

Context

\todo*{mention \gls{IoT} problems ( \gls{DDoS} and botnets), to emphasis usefulness of connected but not online}

Restatement of the problem

Restatement of the response

Roadmap

	\section{Case Study: "The Hygiene Games"} \label{Case Study: "The Hygiene Games"}

		\paragraph{Gamification}
	
		\paragraph{Wireless Networks in Hospitals}

\FloatBarrier