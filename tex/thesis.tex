\documentclass[12pt,a4paper]{report}

% packages
\usepackage[lmargin={3.0cm},rmargin={2.0cm},tmargin={2.5cm},bmargin={2.0cm}]{geometry}
\usepackage[english]{babel}
\usepackage{amssymb}
\usepackage{graphicx}
\usepackage{tikz} % for positioning the logo
\usepackage[latin1]{inputenc}
\usepackage[backend=bibtex,style=numeric,citestyle=numeric]{biblatex} % for references with footnotes
%\usepackage[nottoc,notlot,notlof]{tocbibind} % for inclusion of references in toc
\usepackage[titletoc]{appendix} % for inclusion of references in toc
\usepackage{setspace} % for spacing

\bibliography{references}

% packages only for demonstration/testing
\usepackage{lipsum} % for lorem ipsum



%\documentclass[12pt,a4paper]{report}
\usepackage{graphicx}
\usepackage{tikz} % for positioning the logo
\usepackage[latin1]{inputenc}

\begin{document}
	\begin{titlepage}
	\begin{tikzpicture}[remember picture,overlay]
	\node[anchor=north east,inner sep=10pt] at (current page.north east){\includegraphics[width=1.41cm,keepaspectratio]{figures/FHLogoFromSVG.png}};
	\end{tikzpicture}
	\centering
	{\LARGE Fachhochschule Aachen \par}
	\vspace{0.3cm}
	{\Large Campus J\"ulich \par}
	\vspace{1cm}
	{\Large Fachbereich: Medizintechnik und Technomathematik\par}
	\vspace{0.3cm}
	{\Large Studiengang: Technomathematik\par}
	\vspace{1.5cm}
		{\huge Secure Multi-Party Computation for Decentralized Distributed Systems\par}
	\vspace{1.5cm}
	{\Large Masterarbeit von Frederic Klein\par}
	\vfill
	{\large Diese Arbeit wurde betreut von:\par}
	\vspace{0.3cm}
	{\large
		\begin{tabular}{lr}
			1. Pr\"ufer: & Prof. Dr. rer. nat. Alexander \textsc{Vo}\ss{}\\
			2. Pr\"ufer: & Dr. Stephan \textsc{Jonas}
		\end{tabular}		
	}
	\vfill
	% Bottom of the page
	{\large Aachen, Oktober, 2016 \par}
\end{titlepage}
\end{document}

\begin{document}

\newgeometry{left=3cm}
	\begin{titlepage}
	\begin{tikzpicture}[remember picture,overlay]
	\node[anchor=north east,inner sep=10pt] at (current page.north east){\includegraphics[width=1.41cm,keepaspectratio]{figures/FHLogoFromSVG.png}};
	\end{tikzpicture}
	\centering
	{\LARGE Fachhochschule Aachen \par}
	\vspace{0.3cm}
	{\Large Campus J\"ulich \par}
	\vspace{1cm}
	{\Large Fachbereich: Medizintechnik und Technomathematik\par}
	\vspace{0.3cm}
	{\Large Studiengang: Technomathematik\par}
	\vspace{1.5cm}
		{\huge Secure Multi-Party Computation for Decentralized Distributed Systems\par}
	\vspace{1.5cm}
	{\Large Masterarbeit von Frederic Klein\par}
	\vfill
	{\large Diese Arbeit wurde betreut von:\par}
	\vspace{0.3cm}
	{\large
		\begin{tabular}{lr}
			1. Pr\"ufer: & Prof. Dr. rer. nat. Alexander \textsc{Vo}\ss{}\\
			2. Pr\"ufer: & Dr. Stephan \textsc{Jonas}
		\end{tabular}		
	}
	\vfill
	% Bottom of the page
	{\large Aachen, Oktober, 2016 \par}
\end{titlepage}
\restoregeometry

\pagenumbering{roman}

\clearpage
\vspace*{\fill}
\begin{center}
	\begin{minipage}{.8\textwidth}
		\thispagestyle{empty} % no pagenumber on affidavit
		Diese Arbeit ist von mir selbst\"andig angefertigt und verfasst. Es sind keine anderen als die angegebenen Quellen und Hilfsmittel benutzt worden.\par
		\vspace{1cm}
		Frederic Klein \dotfill \par
		Unterschrift
	\end{minipage}
\end{center}
\vfill % equivalent to \vspace{\fill}
\clearpage

\spacing{1.5} % line spacing

\begin{abstract}
	\thispagestyle{plain}
	\setcounter{page}{1}
	\todo{~1 page}
\end{abstract}

\tableofcontents
\setcounter{page}{2}
\listoffigures
{\let\clearpage\relax \listoftables}

\printnoidxglossary[type=\acronymtype,title={List of Acronyms}]

{\let\clearpage\relax \printnoidxglossary[type=symbols,sort=letter]}

\chapter{Introduction}
\pagenumbering{arabic}
In the last couple of years gamification has found it's way into many areas of our daily life. In regard to our personal life, companies like Amazon or Runtastic can base their gamification approach on publicly sharing personal achievements and statistics to improve user commitment. Gamification concerning our work life on the other hand can have much higher privacy demands. Since comparison is a key component for the gamification approach, privacy protecting computations of system wide statistical values (for example minimum and maximum) are needed. The solution comes in the form of \gls{SMPC}, a subfield of cryptography.

Existing frameworks for \gls{SMPC} utilize the Internet protocol, though access to the Internet or even a \gls{LAN} cannot be provided in all environments. Especially many hospitals tend to avoid Wi-Fi to reduce the risk of electromagnetic interference with medical devices.

To be able to utilize \gls{SMPC} in environments with Wi-Fi restrictions, this thesis studies the characteristics of mesh-networks and proposes describes the design of a \gls{SMPC} framework for mesh-networks. 

Context

Restatement of the problem

Restatement of the response

Roadmap

\chapter{Foundation}

	\section{Case Study: "The Hygiene Games"}

		\subsection*{Gamification}

		\subsection*{Wireless Networks in Hospitals}
	
	\section{Secure Multi-Party Computation}

		\subsection*{Secure Addition Protocol}

		\subsection*{Secure Comparison Protocol}

		\subsection*{Differential Privacy}

		\subsection*{Existing Frameworks}
	
	\section{Mobile Ad Hoc Networks}
	
	{\color{gray} 
		\begin{itemize}  
			\item continuously self-configuring
			\item self-forming
			\item self-healing
			\item infrastructure-less
			\item peer-to-peer
			\item Difference to mesh: mobility of nodes
		\end{itemize}
	}
	
		\subsection*{Smart Phone Ad Hoc Network}
		\todo{Example: firechat}
		
			\subsubsection*{Comparison to Wi-Fi Direct}
			
			{\color{gray} 
				\begin{itemize}  
					\item SPAN support multi-hop relays
					\item Wi-Fi Direct since Android 4.0
					\item Wi-Fi Direct: Soft AP 
				\end{itemize}
			}
		
			\subsubsection*{Wi-Fi Based \gls{SPAN}}
	
			\subsubsection*{Bluetooth Based \gls{SPAN}}
			
	\section{Distributed Computing}
	
		\subsection*{Coordinator Election}
	

\chapter{Methodology and Implementation}

%	\section{Applicability of \gls{SMPC} Protocols in Decentralized Systems}
	
%	\subsection*{Analysis of Key Factors: Computing Power, Network Data Rates and Duration of Connection}
	
%	\section{Effectiveness of \gls{SMPC} Protocols in Sparse Networks}
	
%	\subsection*{Maintaining anonymity}
	
%	\subsection*{Strategies for Aggregation of Participants in Sparse Networks}
	
%	\section{Applicability and Requirements Analysis for the Hygiene Games}
	
\chapter{Evaluation}



\chapter{Conclusion}

\clearpage
\renewcommand{\bibname}{References} % rename Bibliograpy to References
\printbibliography[heading=bibintoc] % add References chapter and display it in toc

\begin{appendices}
	\chapter{Some name}
	\lipsum[3]
\end{appendices}

\end{document}